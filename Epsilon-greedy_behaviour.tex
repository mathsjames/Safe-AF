\documentclass[a4paper]{article}

\usepackage{amsmath}
\usepackage{amsfonts}
\usepackage{mathtools}
\usepackage{pdflscape}
\usepackage{changepage}

\newcommand{\ep}{\langle p \rangle}
\newcommand{\pp}{\langle p^2 \rangle}
\newcommand{\sdp}{\mathrm{sd}(p)}
\newcommand{\ord}{\mathcal{O}}

\setlength{\parindent}{0pt}
\setlength{\parskip}{1em}

\begin{document}

Assumptions:
\begin{itemize}
	\item If the agent exicutes $\epsilon$-greedy with constant $\epsilon$, the environment does not need to be continious
	\item If the agent exicutes a policy that converges to $\epsilon$-greedy with constant $\epsilon$ in the limit $t\rightarrow\infty$ then the environment needs to be continious around $p_i=\epsilon$ and $p_i=1-\epsilon$
\end{itemize}


<<<<<<< HEAD
\textbf{If the agent behaves like $\epsilon$-greedy, for at fixed $\epsilon>0$, in the limit $t\rightarrow\infty$} then the set of explored action probabilities will eventually be dominated by values $p_i$ such that $p_i\in\{\epsilon,1-\epsilon\}$.
=======
\textbf{If the agent behaves like $\epsilon$-greedy, with convergent $\epsilon<1/2$, in the limit $t\rightarrow\infty$} then the set of explored action probabilities will eventually be dominated by values $p_i$ such that $p_i\in\{\epsilon,1-\epsilon\}$.
>>>>>>> 89ac1f4db86b4428e517ce69379fece42adaa1cd

For some $q$, we have that
\begin{equation}
	\ep = q(1-\epsilon) + (1-q)\epsilon = q - 2q\epsilon + \epsilon
\end{equation}
\begin{align}
	\pp &= q(1-\epsilon)^2 + (1-q)\epsilon^2 = q - 2q\epsilon + \epsilon^2\\
	    &= \ep - \epsilon(1-\epsilon)
\end{align}
This allowes us to simplify the expected learned utility difference, $\Delta$ (see table) as a function of only $\ep$ and the constants of the problem. Note that because $\ep$ is dominated by $p_i\in\{\epsilon,1-\epsilon\}$, we must get $\epsilon\leq\ep\leq(1-\epsilon)$

\textbf{Becasue of law ot large numbers (?)}, $\Delta$ will eventual eventually describe the agents belif arbitary well. Let $\Delta=Q(a_1)-Q(a_2)$ and $p_i = P(a_1|t=i)$. If currently $\Delta>0$, the agent will exicute $p_i=1-\epsilon$ and drive $\ep$ towards this value. If currently $\Delta<0$, the agent will exicute $p_i=\epsilon$ and drive $\ep$ towards that value. Therefore the only posible stable points are
\begin{equation}\label{stable_min}
	\ep = \epsilon \quad \& \quad \Delta\leq0
\end{equation}
\begin{equation}\label{stable_max}
	\ep = 1-\epsilon \quad \& \quad \Delta\geq0
\end{equation}
\begin{equation}\label{stable_root}
	\Delta=0 \quad \& \quad \frac{d\Delta}{d\ep}\leq0
\end{equation}
(\ref{stable_min}) and (\ref{stable_max}) represent constant $p_i$ at ether extreme, $\epsilon$ or $1-\epsilon$. (\ref{stable_root}) represent fluctuating $p_i$ but with a stable $\ep$ in between the two extremes.

<<<<<<< HEAD
\textbf{Concider an environment with concists of the agent playing a game agianst it self} with an arbitary payof matrix. The copy will always use the same action probabilites as the agent, for any given round.

=======
\textbf{Concider an agent playing a game with an arbitary payof matrix, agianst a copy of it self.} The copy will always use the same action probabilites as the agent, for any given round.

\begin{center}
>>>>>>> 89ac1f4db86b4428e517ce69379fece42adaa1cd
\begin{tabular}{cc}
	& copy\\
	self
	& \begin{tabular}{c|c|c|}
		& $a_1$ & $a_2$ \\
		\hline
		$a_1$ & $M_{11}$ & $M_{12}$ \\
		\hline
<<<<<<< HEAD
		$a_2$ & $M_{12}$ & $M_{12}$ \\
		\hline
	\end{tabular}
\end{tabular}

This generalisation covers, for example, \textit{Prisoners' dilemma agains copy} and \textit{Death in Damaskus} but not \textit{Absent minded driver} and \textit{Evidential blackmail}.


=======
		$a_2$ & $M_{21}$ & $M_{22}$ \\
		\hline
	\end{tabular}
\end{tabular}
\end{center}

This generalisation covers, for example, \textit{Prisoners' dilemma agains copy} and \textit{Death in Damaskus} but not \textit{Absent minded driver} and \textit{Evidential blackmail}.

If $p_i\in\{\epsilon,1-\epsilon\}$, then
\begin{equation}
	\Delta = M_{11} - M_{22} 
	+ \left(\frac{M_{12}-M_{11}}{\ep} - \frac{M_{21}-M_{22}}{1-\ep}\right)
	\epsilon(1-\epsilon)
\end{equation}
\begin{equation}
	\frac{d\Delta}{d\ep} = 
	\left(\frac{M_{11}-M_{12}}{\ep^2} - \frac{M_{21}-M_{22}}{(1-\ep)^2}\right)\epsilon(1-\epsilon)
\end{equation}
The roots of this expression are
\begin{equation}
	\Delta = 0 \implies 
	\begin{cases}
		\ep = \dfrac{M_{11}-M_{12}}{M_{11}-M_{22}}\epsilon 
			  + \ord(\epsilon^2)\\
		\quad \textbf{or}\\
		1 - \ep = \dfrac{M_{21}-M_{22}}{M_{11}-M_{22}}\epsilon
			  	  + \ord(\epsilon^2)
	\end{cases}
\end{equation}
The derivatives at those points are
\begin{equation}
	\left.\frac{d\Delta}{d\ep}\right|_{\ep = \frac{M_{11}-M_{12}}{M_{11}-M_{22}}\epsilon}
	= \frac{(M_{11}-M_{22})^2}{M_{11}-M_{12}} + \ord(\epsilon)
\end{equation}
\begin{equation}
	\left.\frac{d\Delta}{d\ep}\right|_{1-\ep = \frac{M_{21}-M_{22}}{M_{11}-M_{22}}\epsilon}
	= \frac{(M_{11}-M_{22})^2}{M_{22}-M_{21}} + \ord(\epsilon)
\end{equation}

\textbf{If $\epsilon\rightarrow0$ when $t\rightarrow\infty$, then} 
\begin{equation}
	\ep \text{ can converge to 0} \quad\textbf{iff}\quad (M_{11}<M_{22})\textbf{or}(M_{12}<M_{22})
\end{equation}
\begin{equation}
	\ep \text{ can converge to 1} \quad\textbf{iff}\quad (M_{22}<M_{11})\textbf{or}(M_{21}<M_{11})
\end{equation}
>>>>>>> 89ac1f4db86b4428e517ce69379fece42adaa1cd


\end{document}